\chapter{Introduction to Gravitational Lensing}

Galaxies and clusters of galaxies that act as gravitational lenses can be approximated by single isothermal spheres. It is easy to relate an angular scaling parameter $\xi_{E}$, referred to as the Einstein radius, to the mass inside the corresponding light cone. The Einstein radius corresponds to the ring image of a point source aligned exactly on the axis of the lens.

Summary of isothermal sphere:

\begin{equation}
\rho(r)=\frac{\sigma^2}{2\pi Gr^2}
\end{equation}

\begin{equation}
\Sigma(\xi)=\frac{\sigma^2}{2G\xi}
\end{equation}

\begin{equation}
\xi_{E}=4\pi\left(\frac{\sigma}{c}\right)^{2}\frac{D_{ds}}{D_{s}}
\end{equation}

In reality, the density profile and lensing properties of galaxies is a bit more complicated than the assumption of a singular isothermal sphere, so we need to take into account more complex but elaborate profiles such as the NFW (Navarro, Frenk, White, 1996).

The NFW density profile is 

\begin{equation}
\rho(r)=\frac{\delta_{c}\rho_{c}}{(r/r_{s})(1+r/r_{s})^{2}}
\end{equation}

where the characteristic over density (dimensionless quantity) is given by:

\begin{equation}
\delta_{c}=\frac{200}{3}\frac{c^{3}}{\ln{(1+c)}-c/(1+c)}
\end{equation}

The mass of an NFW halo contained within a radius of $r_{200}$ is:

\begin{equation}
M_{200}=M(r_{200})=\frac{800\pi}{3}\rho_{c}r^{3}_{200}=\frac{800\pi}{3}\frac{\bar{\rho}(z)}{\Omega(z)}r^{3}_{200}
\end{equation}

The concentration parameter $c$ is strongly correlated with Hubble type, c=2.6 separating early from late-type galaxies. Those galaxies with concentration indices $c>2.6$ are early-type galaxies reflecting the fact that the light is more concentrated towards their centres, its formal definition in terms of the virial and characteristic radius is:

$c=r_{200}/r_{s}$

Dutton and Maccio 2014 (in continuation of previous studies such as Mu\~noz Cuartas et. al.), made simulations of halo masses from dwarf galaxies to galaxy clusters and find constraints on the concentration parameter for different redshifts, the relation between the concentration parameter with redshift and virial mass is shown in the following figure:

\begin{figure}[H]
\centering
\includegraphics[width=12cm]{images/dutton.png}
\caption[Evolution of the concentration mass relation]{Evolution of the concentration mass relation, by Dutton \& Maccio, 2014}
\end{figure}

The surface mass density in the NFW profile is given by:

\begin{equation}
\Sigma_{\text{NFW}}(x) = \left\lbrace
\begin{array}{lll}
\frac{2r_{s}\delta_{c}\rho_{c}}{\left(x^{2}-1\right)}\left[1-\frac{2}{\sqrt{1-x^{2}}}\arctanh\sqrt{\frac{1-x}{1+x}}\right](x<1) & (x<1)\\\\
\frac{2r_{s}\delta_{c}\rho_{c}}{3}(x=1) & (x=1)\\\\
\frac{2r_{s}\delta_{c}\rho_{c}}{\left(x^{2}-1\right)}\left[1-\frac{2}{\sqrt{x^{2}-1}}\arctan\sqrt{\frac{x-1}{1+x}}\right](x<1) & (x>1)
\end{array}
\right.
\end{equation} 

so from the critical density:

\begin{equation}
\rho_{c}=\frac{3H^2(z)}{8\pi G}
\end{equation}

$H(z)=H_{0}(1+\Omega z)^{3/2}$





 
 