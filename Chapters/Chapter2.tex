\chapter{Theoretical Framework}


The plot of the enclosed mass shows what radial scale we need to prove. One way could be through dynamics, another could be through gravitational lensing (talk about lensing). If we take different IMFs, how sensitive is the matter content to the choice of IMF?. Smaller the dark matter contribution, the smaller the overall error you make since light is easier to constraint and thus the IMF. Strong lensing measures exactly the enclosed mass so we need to know how much of its contribution we need to substract, the less we have to substract, the better for the determination of the IMF. If the effect of the IMF is very subtle in the mass vs radius plot, then we would need to know the dark matter distribution very well, but if the effect of the IMF is not very subtle, the less you need to know about the dark matter distribution. A recent study of a BCG mentions the relevance of this spatial scale, at very small radii stars dominate the lensing mass, so that lensing provides a direct probe of the stellar mass-to-light ratio, with only small corrections needed for dark matter (Russell Smith and John R. Lucey 2013) 

That's why the IMF is relevant, because it allows to see how much it moves up and down. If you look at a galaxy, why is it not possible to get the mass to light ratio? This is because the dark matter is more diluted than stellar light so the mass follows light behavior is not valid and a well understood theory of the dark matter halos has to be taken into account, NFW (Navarro, Frenk \& White,1996) provided a very consistent model for dark matter halos using N-body simulations so we can relate the lensing of the halos given this NFW density profile and putting special attention in the spacial scales on which the dark matter is relevant and where it starts to be the dominant contribution of the system and thus the lensing.

\begin{figure}[H]
\centering
\includegraphics[width=12cm]{images/galaxies_per_arcmin.png}
\caption[Galaxy Cluster MACS 1206]{Galaxy Cluster MACS 1206, credits to NASA Hubble Space Telescope}
\end{figure}


\section{Stellar populations}
 
Glas.

dwarf stars contribute very little to the integrated light from an old stellar population (Smith 2015)

Galaxy clusters contain a population of stars gravitationally unbound to individual galaxies, yet still bound to the clusters overall gravitational potential, created by the stripping of stars from galaxies during interactions and mergers

\begin{figure}[H]
\centering
\includegraphics[width=12cm]{images/GC.jpg}
\caption[Galaxy Cluster MACS 1206]{Galaxy Cluster MACS 1206, credits to NASA Hubble Space Telescope}
\end{figure}

Quoted (need to change this):The image of galaxy cluster MACS J1206.2-0847 (or MACS 1206) is part of a broad survey with NASA Hubble Space Telescope.The distorted shapes in the cluster are distant galaxies from which the light is bent by the gravitational pull of an invisible material called dark matter within the cluster of galaxies. This cluster is an early target in a survey that will allow astronomers to construct the most detailed dark matter maps of more galaxy clusters than ever before.These maps are being used to test previous, but surprising, results that suggest that dark matter is more densely packed inside clusters than some models predict. This might mean that galaxy cluster assembly began earlier than commonly thought.

Scientists are planning to observe a total of 25 galaxy clusters under a project called CLASH (Cluster Lensing and Supernova survey with Hubble).One of the first objects observed for the new census is the galaxy cluster MACS J1206.2-0847. This conglomeration of galaxies is one of the most massive structures in the universe, and its gigantic gravitational pull causes stunning gravitational lensing. MACS 1206 lies 4 billion light-years from Earth. In addition to curving of light, gravitational lensing often produces double images of the same galaxy. In the new observation of cluster MACS J1206.2-0847, astronomers counted 47 multiple images of 12 newly identified galaxies.The era when the first clusters formed is not precisely known, but is estimated to be at least 9 billion years ago and possibly as far back as 12 billion years ago. If most of the clusters in the CLASH survey are found to have excessively high accumulations of dark matter in their central cores, then it may yield new clues to the early stages in the origin of structure in the universe.


\section{Gravitational Lensing}

Galaxies and clusters of galaxies that act as gravitational lenses can be approximated by single isothermal spheres. It is easy to relate an angular scaling parameter $\xi_{E}$, referred to as the Einstein radius, to the mass inside the corresponding light cone. The Einstein radius corresponds to the ring image of a point source aligned exactly on the axis of the lens.

Summary of isothermal sphere:

\begin{equation}
\rho(r)=\frac{\sigma^2}{2\pi Gr^2}
\end{equation}

\begin{equation}
\Sigma(\xi)=\frac{\sigma^2}{2G\xi}
\end{equation}

\begin{equation}
\xi_{E}=4\pi\left(\frac{\sigma}{c}\right)^{2}\frac{D_{ds}}{D_{s}}
\end{equation}

In reality, the density profile and lensing properties of galaxies is a bit more complicated than the assumption of a singular isothermal sphere, so we need to take into account more complex but elaborate profiles as the NFW (Navarro, Frenk, White, 1996).

The NFW density profile is 

\begin{equation}
\rho(r)=\frac{\delta_{c}\rho_{c}}{(r/r_{s})(1+r/r_{s})^{2}}
\end{equation}

where the characteristic over density (dimensionless quantity) is given by:

\begin{equation}
\delta_{c}=\frac{200}{3}\frac{c^{3}}{\ln{(1+c)}-c/(1+c)}
\end{equation}

The mass of an NFW halo contained within a radius of $r_{200}$ is:

\begin{equation}
M_{200}=M(r_{200})=\frac{800\pi}{3}\rho_{c}r^{3}_{200}=\frac{800\pi}{3}\frac{\bar{\rho}(z)}{\Omega(z)}r^{3}_{200}
\end{equation}

For A754 we take a $M_{200}=9.8\times 10^{15} M\odot$ (Sifon et. al. 2015)

The concentration parameter $c$ is strongly correlated with Hubble type, c=2.6 separating early from late-type galaxies. Those galaxies with concentration indices $c>2.6$ are early-type galaxies reflecting the fact that the light is more concentrated towards their centres, its formal definition in terms of the virial and characteristic radius is:

$c=r_{200}/r_{s}$

Dutton and Maccio 2014 (in continuation of previous studies such as Munoz Cuartas et. al.), made simulations of halo masses from dwarf galaxies to galaxy clusters and find constraints on the concentration parameter for different redshifts, the relation between the concentration parameter with redshift and virial mass is shown in the following figure:

\begin{figure}[H]
\centering
\includegraphics[width=12cm]{images/dutton.png}
\caption[Evolution of the concentration mass relation]{Evolution of the concentration mass relation, by Dutton \& Maccio, 2014}
\end{figure}

let's take the case of ABELL1068, it's magnitude in U is 21.94, in I is 18.46, in g is 20.09, in r is 19.5, also $M_{200}=4.3\times 10^{14}M_{\odot}$ (van der Burg et. al 2015)

The bolometric luminosity of Abell1068 is $10^{44}$erg/s that in solar luminosities is $1.9\times 10^{12} L_{\odot}$, this gives an effective brightness of $0.962\times 10^{7}M_{\odot}/kpc^2$.

the distance to the galaxy is 591.42857 Mpc

In the case of the ABELL1068 cluster, our estimation yields a concentration parameter of 4.46.

The surface mass density in the NFW profile is given by:

\begin{equation}
\Sigma_{\text{NFW}}(x) = \left\lbrace
\begin{array}{lll}
\frac{2r_{s}\delta_{c}\rho_{c}}{\left(x^{2}-1\right)}\left[1-\frac{2}{\sqrt{1-x^{2}}}\arctanh\sqrt{\frac{1-x}{1+x}}\right](x<1) & (x<1)\\\\
\frac{2r_{s}\delta_{c}\rho_{c}}{3}(x=1) & (x=1)\\\\
\frac{2r_{s}\delta_{c}\rho_{c}}{\left(x^{2}-1\right)}\left[1-\frac{2}{\sqrt{x^{2}-1}}\arctan\sqrt{\frac{x-1}{1+x}}\right](x<1) & (x>1)
\end{array}
\right.
\end{equation} 

then the concentration parameter for ABELL1068 is about 7.9 supposing a mass of the galaxy of $10^{12.5}M_{\odot}$

so from the critical density:

\begin{equation}
\rho_{c}=\frac{3H^2(z)}{8\pi G}
\end{equation}

The critical density would be: $2\times 10^{-26}$ in SI units so in $M_{\odot}/pc^{3}$ it is $2.9\times 10^{-7}$

$H(z)=H_{0}(1+\Omega z)^{3/2}$

the Hubble parameter at z=0.138 is H(z)=85.6

The characteristic radius is given by $r_{1/2}=1.34R_{e}$

For the stellar content of the cluster we can use de Vaucouleurs law for the surface brightness distribution in giant elliptical galaxies which is:

\begin{equation}
I(R)=I_{e}e^{-b\left[\left(R/R_{e}\right)^{1/4}-1\right]}
\end{equation}

where $b=7.67$ and $I_{e}$ is the effective brightness which is basically the brightness at the effective radius $R_{e}$

From the paper of Lokas and Mamon, for constant mass-to-light-ratio we have $\Sigma_{M}(R)= \Upsilon I(R)$ where $I(R)\approx 10^{7}$ was found by fitting the surface brightness with \textit{GALFIT}.

The mass to light ratio is $\Upsilon\approx 4$

Hence we have the surface mass density for both the stellar content and the NFW profile:

\begin{figure}[H]
\centering
\includegraphics[width=12cm]{images/Surface_mass_density_log.png}
\caption[Surface mass density profiles]{Surface mass density profiles in logarithmic and $R^{1/4}$ scale for the NFW profile and the stellar component.}
\end{figure}

But we are more interested in the enclosed mass which can be done by integrating the surface mass density:

\begin{equation}
M(R)=\int_{0}^{R}2\pi R\Sigma(R)dR
\end{equation}

And we can recover our luminosity by integrating the surface brightness profile accordingly:

\begin{equation}
L=\int_{0}^{R}2\pi RI(R)dR
\end{equation}

The integration gives a value that is comparable to the one found using Faber-Jackson relation: $L=\Upsilon\times\sigma^{4}\approx 1.2\times 10^{12}M_{\odot}$

The plot for the enclosed mass is:

\begin{figure}[H]
\centering
\includegraphics[width=12cm]{images/DM_fraction.png}
\caption[Enclosed mass and DM to stellar mass ratio]{Enclosed mass and DM to stellar mass ratio}
\end{figure}

The value found for the mass in light is $M_{\star}=2.582\times 10^{11}M_{\odot}$ and the mass given by the NFW profile is $M_{\text{NFW}}=6.557\times 10^{11}M_{\odot}$.

Now, we are interested in having an accurate estimate of the Einstein radius to constraint the model, so we make different analysis on the radial dependence on the lensing properties such as shear, reduced shear and magnification.

The radial dependence on the shear is:

\begin{equation}
\gamma_{\text{NFW}}(x) = \left\lbrace
\begin{array}{lll}
\frac{r_{s}\delta_{c}\rho_{c}}{\Sigma_c}g_{<}(x) & (x<1)\\\\
\frac{r_{s}\delta_{c}\rho_{c}}{\Sigma_c}\left[\frac{10}{3}+4 \ln \left(\frac{1}{2}\right)\right] & (x=1)\\\\
\frac{r_{s}\delta_{c}\rho_{c}}{\Sigma_c}g_{>}(x) & (x>1)
\end{array}
\right.
\end{equation} 

where: 

\begin{equation}
g_{<}(x)=\frac{8 \arctanh \sqrt{\frac{1-x}{1+x}}}{x^{2}\sqrt{1-x^{2}}}+\frac{4}{x^{2}} \ln \left(\frac{x}{2}\right)-\frac{2}{\left(x^{2}-1\right)}+\frac{4 \arctanh \sqrt{\frac{1-x}{1+x}}}{\left(x^{2}-1\right)\left(1-x^{2}\right)^{1/2}}
\end{equation}

\begin{equation}
g_{<}(x)=\frac{8 \arctan \sqrt{\frac{x-1}{1+x}}}{x^{2}\sqrt{x^{2}-1}}+\frac{4}{x^{2}}\ln \left(\frac{x}{2}\right)-\frac{2}{\left(x^{2}-1\right)}+\frac{4 \arctan \sqrt{\frac{x-1}{1+x}}}{\left(x^{2}-1\right){}^{3/2}}
\end{equation} 

and with the critical surface mass density:

\begin{equation}
\Sigma_{c}\equiv\frac{c^{2}}{4\pi G}\frac{D_{s}}{D_{d}D_{ds}}
\end{equation}

these equations come from the paper Wright and Brainerd 1999

the plot of the shear dependence on the radius is:

\begin{figure}[H]
\centering
\includegraphics[width=12cm]{images/Shear_vs_rad.png}
\caption[Shear dependence on radius]{Shear dependence on radius for different redshift of the background galaxies}
\end{figure}

The magnification tensor is:

\begin{equation}
\frac{\partial\beta}{\partial\theta}=\delta_{ij}-\frac{\partial^{2}\psi}{\partial\theta_{i}\partial\theta_{j}}=\left(\begin{array}{cc}
1-\kappa-\gamma_{1} & -\gamma_{2}\\
-\gamma_{2} & 1-\kappa+\gamma_{1}
\end{array}\right)
\end{equation}

The total magnification $\mu$ is given by the determinant of the magnification tensor:

\begin{equation}
\mu = \frac{1}{(1-\kappa)^{2}-\gamma^{2}_{1}-\gamma^{2}_{2}}
\end{equation}


Where $\kappa$ is the convergence that determines the magnification and $\gamma_{1}$ and $\gamma_{2}$ are the shear components that determine the distortion of the background objects.

The magnification is then:

\begin{figure}[H]
\centering
\includegraphics[width=12cm]{images/Magnification.png}
\caption[Magnification radial profile]{Magnification radial profile for various redshifts}
\end{figure}

The reduced shear is given by:

\begin{equation}
g=\frac{\gamma}{1-\kappa}
\end{equation}

The reduced shear for background objects at different redshifts is: 

\begin{figure}[H]
\centering
\includegraphics[width=12cm]{images/Reduced_Shear.png}
\caption[Reduced shear radial]{Reduced shear radial profile for different redshifts.}
\end{figure}

so we get the Einstein ring where $\mu$ is infinite or when g is 1 (k=1/2)

\section{IMF in BCGs}

number of stars per unit mass

Kroupa, Chabrier, Salpeter, 

Heavyweight

It's difficult to see how much of the faint stars contribute to the mass of the system. We only see the new bright ones

BCG -> giant ellipticals

For stars, measurements of the luminosity function can be used to derive the Initial Mass Function (IMF). For galaxies, this is more difficult because Mass to light ratio (M/L) of the stellar population depends upon the star formation history of
the galaxy. Bulges have heavier IMFs than disks

Several recent studies have presented evidence for "heavyweight" IMFs in giant ellipticals, with a mass-to-light-ratio twice that of a Milky Way like IMF.



 
 