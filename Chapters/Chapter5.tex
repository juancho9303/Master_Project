\chapter{Conclusions}

The COSMOS experiment shows that we should expect a few lensed objects in the studied spatial scale and with the sensitivity of our data. Even though the number is small, a proper subtraction of the BCG allows us to see a few objects in the innermost region of the clusters. In our sample we indeed find a few background galaxies (using the photo-$z$'s) but none of them are suitable lensing candidates.

Perhaps the most important conclusion in this work is that for the purpose of characterizing IMFs in massive systems, it is better to use galaxies than BCGs because the stellar contribution is less important in the latter, even in the small radial distances from the centre. The overwhelming dominance of DM in the centre of galaxy clusters makes the determination of the stellar content harder since we can't assume that mass follows light thus lensing doesn't provide direct information about the baryonic mass. In the case of galaxies, even though the spatial scales in which stellar content is important are also small, they allow for a better constraint using gravitational lensing since these distances are close to the Einstein rings.

Except for the long arc in Abell 1413, we don't have evidence for strong lensing candidates in our sample. This cluster is a good candidate for follow up studies. Unfortunately, in our sample we don't have $U$ band data for this cluster so the characterization of the lens using photometric redshifts couldn't be done properly. 

Another interesting result in our observational procedures is that after the subtraction of the BCGs (at least in our sample) the inner regions don't show a general trend. Some cluster centres appear to have a more intense dynamical state and are very crowded with cluster galaxies. On the other hand, some clusters are much more empty in their inner regions.

One of the most complicated procedures in our observational analysis was the photometry in the clusters after the subtraction of the BCG because the subtraction of the most prominent galaxies always affects the background. The careful background determination and precise photometry is not trivial and require very detailed assumptions in every case since only a good photometry yields trustworthy photometric redshifts. This detailed analysis would be relevant in follow up studies.

The spatial scale that we study here (the innermost region of galaxy clusters centred at the center of the BCGs) is very small, so the Einstein rings that are in that region are the ones associated to low redshift objects (as shown in Figures [3.8] and [3.9]). This means that the lensed objects that we expect to see in these systems would most likely be low redshift galaxies.

\section{Possible future improvements}

Follow up studies like \textcolor{blue}{Smith et al.} (\citeyear{Reference34}) should make use of spectroscopic data to properly get the redshifts of the background objects and compare them to the photo-$z$'s. This way, the determination of the lensing candidates can be done in a more precise way.

Since photometric redshift determination is a complicated procedure and it strongly depends on the filters used, it is important to have more filters than the ones we used for this work. The use of more filters would not only give better colour images but also would give more accurate photo-$z$'s. 

It is also important to take into account that the fitting of the BCG should include various components such as a central mass that affects not only the mass of the system but also the brightness profile which ultimately modifies the background determination in the photometry.


\newpage
\lhead{\emph{Appendix}} 
\begin{appendices}
\textbf{{\LARGE Appendix}}

$\qquad$

\textbf{Isothermal Sphere} 

Summary of isothermal sphere:

\begin{equation}
\rho(r)=\frac{\sigma^2}{2\pi Gr^2}
\end{equation}

\begin{equation}
\Sigma(\xi)=\frac{\sigma^2}{2G\xi}
\end{equation}

The Einstein radius:

\begin{equation}
\xi_{E}=4\pi\left(\frac{\sigma}{c}\right)^{2}\frac{D_{ds}}{D_{s}}
\end{equation} 
 
\textbf{NFW profile formalism}
 
The NFW density profile is 

\begin{equation}
\rho(r)=\frac{\delta_{c}\rho_{c}}{(r/r_{s})(1+r/r_{s})^{2}}
\end{equation}

where the characteristic over density (dimensionless quantity) is given by:

\begin{equation}
\delta_{c}=\frac{200}{3}\frac{c^{3}}{\ln{(1+c)}-c/(1+c)}
\end{equation}

The mass of an NFW halo contained within a radius of $r_{200}$ is:

\begin{equation}
\text{M}_{200}=\text{M}(r_{200})=\frac{800\pi}{3}\rho_{c}r^{3}_{200}=\frac{800\pi}{3}\frac{\bar{\rho}(z)}{\Omega(z)}r^{3}_{200}
\end{equation}

The concentration parameter $c$ is strongly correlated with Hubble type, $c=2.6$ separating early from late-type galaxies. Those galaxies with concentration indices $c>2.6$ are early-type galaxies reflecting the fact that the light is more concentrated towards their centres, its formal definition in terms of the virial and characteristic radius is $c=r_{200}/r_{s}$.

\textcolor{blue}{Dutton \& Maccio} (\citeyear{Reference23}) (in continuation of previous studies such as \textcolor{blue}{Mu\~noz Cuartas et al.} \citeyear{Reference12}), made simulations of halo masses from dwarf galaxies to galaxy clusters and find constraints on the concentration parameter for different redshifts, the relation between the concentration parameter with redshift and virial mass is shown in Figure [1].

\begin{figure}[H]
\centering
\includegraphics[width=10cm]{images/dutton.png}
\caption[Evolution of the concentration mass relation]{Evolution of the concentration mass relation, by \textcolor{blue}{Dutton \& Maccio} (\citeyear{Reference23}).}
\end{figure}

The surface mass density in the NFW profile (\textcolor{blue}{Wright \& Brainerd}, \citeyear{Reference4}) is given by:

\begin{equation}
\Sigma_{\text{NFW}}(x) = \left\lbrace
\begin{array}{lll}
\frac{2r_{s}\delta_{c}\rho_{c}}{\left(x^{2}-1\right)}\left[1-\frac{2}{\sqrt{1-x^{2}}}\arctanh\sqrt{\frac{1-x}{1+x}}\right] & (x<1)\\\\
\frac{2r_{s}\delta_{c}\rho_{c}}{3} & (x=1)\\\\
\frac{2r_{s}\delta_{c}\rho_{c}}{\left(x^{2}-1\right)}\left[1-\frac{2}{\sqrt{x^{2}-1}}\arctan\sqrt{\frac{x-1}{1+x}}\right] & (x>1)
\end{array}
\right.
\end{equation} 

so from the critical density:

\begin{equation}
\rho_{c}=\frac{3H^2(z)}{8\pi G}
\end{equation}

$H(z)=H_{0}(1+\Omega z)^{3/2}$

But we are more interested in the enclosed mass which can be done by integrating the surface mass density:

\begin{equation}
\text{M}(R)=\int_{0}^{R}2\pi R\Sigma(R)dR
\end{equation}

The radial dependence on the shear is:

\begin{equation}
\gamma_{\text{NFW}}(x) = \left\lbrace
\begin{array}{lll}
\frac{r_{s}\delta_{c}\rho_{c}}{\Sigma_c}g_{<}(x) & (x<1)\\\\
\frac{r_{s}\delta_{c}\rho_{c}}{\Sigma_c}\left[\frac{10}{3}+4 \ln \left(\frac{1}{2}\right)\right] & (x=1)\\\\
\frac{r_{s}\delta_{c}\rho_{c}}{\Sigma_c}g_{>}(x) & (x>1)
\end{array}
\right.
\end{equation} 

where: 

\begin{equation}
g_{<}(x)=\frac{8 \arctanh \sqrt{\frac{1-x}{1+x}}}{x^{2}\sqrt{1-x^{2}}}+\frac{4}{x^{2}} \ln \left(\frac{x}{2}\right)-\frac{2}{\left(x^{2}-1\right)}+\frac{4 \arctanh \sqrt{\frac{1-x}{1+x}}}{\left(x^{2}-1\right)\left(1-x^{2}\right)^{1/2}}
\end{equation}

\begin{equation}
g_{<}(x)=\frac{8 \arctan \sqrt{\frac{x-1}{1+x}}}{x^{2}\sqrt{x^{2}-1}}+\frac{4}{x^{2}}\ln \left(\frac{x}{2}\right)-\frac{2}{\left(x^{2}-1\right)}+\frac{4 \arctan \sqrt{\frac{x-1}{1+x}}}{\left(x^{2}-1\right){}^{3/2}}
\end{equation}  

For reference, Figure [2] shows strong systematic variation of the IMF in early-type galaxies as a function of their stellar mass-to-light ratio, producing differences up to a factor of three in mass by \textcolor{blue}{Cappellari et al.} (\citeyear{Reference19})

\begin{figure}[H]
\centering
\includegraphics[width=12cm]{images/IMFs_paper.png}
\caption[The systematic variation of the IMF in early-type galaxies.]{The systematic variation of the IMF in early-type galaxies, \textcolor{blue}{Cappellari et al.} (\citeyear{Reference19}).}
\end{figure}
 
\end{appendices}