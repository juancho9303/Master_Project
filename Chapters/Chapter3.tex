%\usepackage{subfig}

\chapter{Observational Procedures}

the full description of the survey is in: D. J. Sand et. al. 2011

MegaCam wide field imager on the CFHT (Canada-France-Hawii Telescope). The cluster sample consisted of 101 clusters within the range of redshifts from 0.05 < z< 0.55

58 clusters from the MENEACs (Multi-Epoch nearby cluster survey)

The meneacs clusters represent all clusters in the BAX X-ray cluster database that are observable for the CFHT

About 60 clusters, but we used only 30 for the final studies and paid special attention to 10, marked with *

G, U, I and R images


\begin{table}[]
\centering

\begin{tabular}{ccccc}
Cluster & $z$   & $\sigma(km/s)$ & $d(Mpc)$ & $\theta_{E}(")$ \\ \hline \hline
A1033   & 0.126 & 762            & 540 & 14.6155  \\
A1068*  & 0.138 & 740            & 591.4 & 13.5945  \\
A1132   & 0.136 & 727            & 582.9 & 13.1515   \\
A119*   & 0.044 & 875            & 188.6 & 21.0798   \\
A1413*  & 0.143 & 881            & 612.9 & 19.1569   \\
A1650   & 0.084 & 720            & 360 & 13.6758   \\
A1651   & 0.085 & 903            & 364.3 & 21.4876   \\
A1795   & 0.062 & 778            & 265.7 & 16.3514   \\
A2029*  & 0.077 & 1152           & 330 & 35.2776   \\
A2050   & 0.118 & 854            & 505.7 & 18.5258   \\
A2055   & 0.102 & 697            & 437.1 & 12.5642   \\
A2064   & 0.108 & 675            & 462.9 & 11.7048   \\
A2065*  & 0.073 & 1095           & 312.9 & 32.0110   \\
A2069   & 0.116 & 966            & 497.1 & 23.7574   \\
A2142*  & 0.091 & 1086           & 390 & 30.8756   \\
A2319*  & 0.056 & 1101           & 240 & 32.9563   \\
A2420   & 0.085 & ~800           & 364.3 & 16.8653   \\
A2440   & 0.091 & 766            & 390 & 15.3608   \\
A2597   & 0.085 & 682            & 364.3 & 12.2569   \\
A2627   & 0.126 & ~800           & 540 & 16.1096   \\
A2703   & 0.114 & ~800           & 488.6 & 16.3307   \\
A399    & 0.072 & ~800           & 308.6 & 17.1049   \\
A553    & 0.066 & ~800           & 282.9 & 17.2155   \\
A655*   & 0.127 & ~800           & 544.3 & 16.0911   \\
A754*   & 0.054 & ~800           & 231.4 & 17.4367   \\
A763    & 0.085 & ~800           & 364.3 & 16.8653   \\
A795    & 0.136 & ~800           & 582.9 & 15.9252   \\
A85*    & 0.055 & ~800           & 235.7 & 17.4182   \\
A961    & 0.124 & ~800           & 531.4 & 16.1464   \\
A990    & 0.144 & ~800           & 617.1 & 15.7778   
\end{tabular}
\caption[Abell Clusters and their redshift]{Abell clusters and their redshifts as given by C. Bildfell et. al. 2012. Marked with * the chosen clusters with the most promising features}
\end{table}

The original images have dimensions of [11000:11000] pixels but since our relevant region is the center of the cluster where the BCG is located, we cut the images with dimension of [1000,1000] for the color analysis and [4000:4000] to characterize the colors and discriminate between cluster and non-cluster members.

The INT images were multiple exposures so it was necessary to make a mosaic of them using SWARP.

\section{Sextractor}

Segmentation image that will be used as a mask image (bad pixels) for GALFIT

We need to discriminate between field stars and the galaxies of the cluster so in order to do this, we used some of the parameters found by \textit{SEXTRACTOR} that allow us to constraint the fitted data. These are class-star, flux\_radius, and FWHM (full wicth half maximum). Class-star uses the neural network star/galaxy of \textit{SEXTRACTOR} that will give values close to 1 for stars and 0 for galaxies. flux\_radius, and FWHM are closely related to each other and give the radius which contains half of the light of the object so it will be small for stars and bigger for extended objects.

In order to extract the same objects and make the segmentation masks for the desired objects in the different filters, we used SEXTRACTOR on dual mode

Color magnitude diagram for A1068

we used a zero point magnitude of 30

\begin{figure}[H]
\centering
\includegraphics[width=12cm]{images/color_mag.png}
\caption[Color Magnitude diagram of A1068]{Color Magnitude diagram of A1068 with the differentiation of stars from galaxies}
\end{figure}

Mag vs flux rad to discriminate

\begin{figure}[H]
\centering
\includegraphics[width=12cm]{images/mag_vs_flux_rad.png}
\caption[Magnitude vs Flux radius of A1068]{Magnitude vs Flux radius of A1068 to identify the galaxies using the criteria of their flux distribution}
\end{figure}

\section{Galfit}

\textit{GALFIT} (Peng et. al) fits two dimensional profiles so it is a useful tool to remove the light from the BCG and allow us to observe background objects

Fit Sersic profiles with n=4 which is de Vaucouleurs profile. 

A first run gives us a rough idea of the true position of the center of the BCG so we can set this values in a second run for each cluster. We needed to combine different Sersic parameters, as well as Fourier and bending modes for some of the BCGs.

We use the segmentation masks given by \textit{SEXTRACTOR} to mask bright objects in the fitting of the BCG

the fitting of many objects (not only the BCG)

the best results were given when we masked the innermost region of the BCG so the fitting will put more weight in the rest of the profile, thus reducing most of the light that hides the background objects.

we have to take into account the magnification bias

The parameters C0, B1, B2, F1, F2, etc. listed below are hidden from the user unless he/she explicitly requests them.  These can  be tagged on to the end of any previous components except, of course, the PSF and the sky -- although \textit{GALFIT} won't bar you from doing so, and will just ignore them.  Note that a Fourier or Bending mode amplitude of exactly 0 will cause \textit{GALFIT} to crash because the derivative image \textit{GALFIT} computes internally will be entirely 0.  If a Fourier or Bending amplitude is set to 0 initially \textit{GALFIT} will reset it to a value of 0.01.  To prevent \textit{GALFIT} from doing so, one can set it to any 
other value.

Bending modes
B1)  0.07      1        Bending mode 1 (shear)
B2)  0.01      1        Bending mode 2 (banana shape)
B3)  0.03      1        Bending mode 3 (S-shape)

Azimuthal fourier modes
F1)  0.07  30.1  1  1   Az. Fourier mode 1, amplitude and phase angle
F2)  0.01  10.5  1  1   Az. Fourier mode 2, amplitude and phase angle
F6)  0.03  10.5  1  1  Az. Fourier mode 6, amplitude and phase angle
F10)  0.08  20.5  1  1   Az. Fourier mode 10, amplitude and phase angle
F20)  0.01  23.5  1  1   Az. Fourier mode 20, amplitude and phase angle

Traditional Diskyness/Boxyness parameter c
C0) 0.1         0       traditional diskyness(-)/boxyness(+)

The masks:

\begin{figure}[H]
\centering
\includegraphics[width=15cm]{images/masks.png}
\caption[Segmentation images]{Segmentation images produced by SEXTRACTOR and used as mask files for the galfit extraction. Left panel is the original mask with all the bright objects. Right panel is the mask after the substraction of the regions surrounding the cluster galaxies to be fitted with GALFIT.}
\end{figure}

The colors are inverted for an easier visualization of the image. The fainter regions are actually the most luminous objects because \textit{GALFIT} assigns increasing numbers starting from the brightest one, that is the BCG in this case

The original image, the fitted models and the output are presented here:

\begin{figure}[H]
\centering
\includegraphics[width=15cm]{images/galfit.png}
\caption[Galfit results]{Galfit procedures. Left: Original image in zscale with the clear BCG expanding across a significant region of the central area. Middle:The models fitted by GALFIT for all the selected cluster galaxies. Right:Residual image after the substraction of the model galaxies.}
\end{figure}

\section{Color images} 

We use IRAF to make the color images using our g,r,u,i bands 

Here we take an isothermal sphere to model the Einstein ring in a distance of background objects of z=1

We made a color image of the original center of the cluster without subtracting the BCG in order to differentiate between cluster members from background galaxies and field stars. This allows us to fit only the cluster galaxies. 

\begin{figure}[H]
\centering
\includegraphics[width=12cm]{images/cA754.jpg}
\caption[Color image of A754]{Color image of A754 cluster (filters i,g,u) with its Einstein radius calculated for an isothermal sphere of a background object at $z=1$.}
\end{figure}

After choosing the galaxies that belong to the cluster by comparing their relative colors, we subtracted them using \textit{GALFIT} and made the color image again changing the scaling values with the task CONVERT of IRAF so that we see can see the color contrast to search for good candidates of lensed objects. By looking at this reduced color image, we have another visual constraint to choose the clusters in which it would be worth to do photometric redshifts and search for objects with the same redshift in different locations around the very center of the BCG (object that has suffered strong lensing). 

\begin{figure}[H]
\centering
\includegraphics[width=12cm]{images/cA754_galfit.jpg}
\caption[Color image of A754 after fitting the bright objects]{Color image of A754 cluster (filters i,g,u) after the substraction of the bright cluster galaxies.}
\end{figure}

Because we have 4 bands we were able to make different color images to see the contrast and make combinations that would allow us to see better the very red and very blue objects, in the following figure we have the g-r, i-r-g and i-g-u color images for three clusters. 

\begin{figure}[H]
\centering
\includegraphics[width=15cm]{images/full_real.jpg}
\caption[Color images for various clusters]{Different color images for different combination of the g,r,u,i filters for the clusters A961, A2703, A1033. Left column for the images constructed only with the g and r filter, central column for i,g,r and right column for i,g,u.}
\end{figure}

\section{Photometric Redshift}

We use the \textit{COSMOS2015} (Laigle et. al 2016.) catalogue that contains photometric redshift of over half million galaxies in multiple bands to put another constraint in our study.

We use the matched catalogue for CFHT by ----- and use the r-band. Our limiting magnitude is 23 in that band so we estimate the number of galaxies per redshift bin that we would expect to see in our sample with that limiting magnitude.

This sets an interesting constraint on what to expect in the inner region of the BCG and give us more information about where to search for good candidates.

The limiting magnitude is found using \textit{SEXTRACTOR} that gives a good confidence on the detection of objects.

(using as reference Benitez, Narciso 2000)

To measure the photometric redshift of the galaxies (after filtering out the stars) in the inner region of the cluster after the subtraction of the BCG, we use the photometric redshift code EAZY (Brammer et. al 2008) which uses an extensive collection of spectral energy distributions for galaxies in the range $0<z<4$. Fortunately, the code includes library from CFHT in the I and U bands but doesn't have the filters in the G and R bands so I used the SUBARU survey filter information to be able to compute the photometric redshifts using four bands.

If you find this code useful, please include a citation to "Brammer, van Dokkum and Coppi, 2008, ApJ, 686, 1503" in the bibliography of any published work that makes use of EAZY.