\chapter{Introduction}

The history of the formation of stars is a key topic in the understanding of galaxies since it determines most of the physical processes of the initial stages and evolution of these building blocks of our Universe. Understanding the way stars form allows us to comprehend many physical properties of their host galaxies thus providing a useful framework on which to build a more elaborate theory of their subsequent evolution. We might have good ideas and some general agreement in the basics of formation of stars in galaxies, but our observational limitations don't allow us to say much about distant objects which we need to make a more elaborate and complete theory. In principle, we can't assume that all the stars have the same formation history in every galaxy and for every epoch of the Universe. The gas clouds that form stars might or might not create the same mixture of stars in every stellar system so it is important to see under what conditions we could assume a general trend and what implications in our observations this may have. 

For galaxies that are far away, it is impossible to make star counts with our current technology, for this reason, their mass-to-light-ratio $\Upsilon$ (given by their stellar populations) provides a simple constraint on their initial mass function (IMF, which is a very fundamental and important quantity in the study of stellar systems because it constraints the physics of star formation but also because it allows us to infer stellar masses through observed luminosities.) as discussed by Russell J. Smith \& John R. Lucey, 2013. Everything we know from galaxy evolution is implicitly assuming an explicit form of the IMF, with very little variations since it is the method we use to connect evolutionary sequences, this of course, given the fact that if every galaxy had its own IMF then it would be too difficult to study their evolution because of the lack of any knowledge about their history. We have some observational information about IMF in galaxies, in the case of spiral galaxies for example, the most commonly used IMFs are Chabrier or Kroupa which are decently constrained given the facilities of our observations in our own galaxy. Also, bulges appear to have heavier IMFs than disks as mentioned by (Dutton et. al 2013), but our current understanding of this topic is still quite far from being satisfactory.

Although these naive assumptions given by our limited observational evidence might not be too far from reality, we must note that when we study more complex and dense systems like the brightest cluster galaxies (BCG) in galaxy clusters or giant elliptical galaxies in general, constraining the IMF via $\textrm{M}_{\star}/\textrm{L}$ might be way more complex and poses a greater challenge since masses are more difficult to establish for dynamically-hot systems like them. Measuring $\Upsilon$ in these systems is not a truly accurate constraint on the IMF since we may have different stellar formation histories than the ones associated with galaxies that are being formed now. These objects have a very old origin (although their build up and morphological formation is recent) because their stellar populations are old and they correspond to the highest density peak, so it is difficult to relate their stellar populations accurately.

This general view shows that in the context of the evolution of galaxies, there are many things that come together at the very heart of cosmology but also in the context of the stellar astrophysics and they need to be consistent with each other. Addressing this problem is complex for many reasons, one of them is that these systems have a strong dependency on their non-baryonic matter content which affects the mass-to-light-ratio determination. This dark matter contribution accounts for most of the dynamical mass of galaxies and it's the dominant contribution in most of their spacial scales, specially in the outter regions. The problem would be much easier to study if we only had the stellar mass because the light measurements would be enough to constrain the stellar populations, their evolution and their mass distribution. 

Being able to calculate the percentage of dark matter allow us to define the IMF more precisely. So we want to see what fraction of the surface density is given by stars and what are the spatial scales in which DM becomes the dominant contribution to the enclosed mass. DM halos seem to have a diluted profile in comparison to the stellar content of galaxies (Navarro Frenk White, ....) so there is a region near the center of these massive systems in which the stellar mass is the dominant contribution. This implies that accurate measurements of their luminosity could give precise determinations of their mass to light ratio thus giving us some knowledge of their IMFs.

Various techniques have been developed to try to understand the stellar populations that form these massive systems. One of them is by using gravitational lensing (Treu et. al. 2010) of background galaxies. Modelling the lensing configuration on a BCG provides a useful method to determine stellar and dark matter mass contribution in elliptical galaxies, since it is difficult to constraint the IMF via $\textrm{M}_{\star}/\text{L}$ as mentioned before. Finding strong lensing in these systems can also give us information about the location of the mass center of the cluster through the lensing they produce. We usually assume that the centre of galaxy clusters lies in the BCGs (George et. al. 2012) but the real position of the centre in galaxy clusters is still an unsolved problem (Harvey et. al, 2017). 

In this project we work with galaxy clusters that might be in the right range to search for gravitational lensing in the inner regions. We use deep data from CFHT that allows us to search for interesting targets and probe the relevant spacial scales. We focus on the brightest cluster galaxy since it is a very massive system that could lens background objects and because photometry measurements can be made very accurately on them in comparison with their neighbouring galaxies. Although removing the BCG with theoretical models is quite difficult, it is still a good target to look at. 
   
\newpage
