\chapter{Introduction}

Stellar mass to light ratio and Stellar populations in the BCGs

Galaxy Clusters

IMF is a very fundamental and important quantity in the study of stellar systems because it constraints the physics of star formation but also because it allows us to infer stellar masses through observed luminosities.

the correct use of an IMF in the context of gravitational lensing on massive objects like early type galaxies in galaxy clusters can help us constraint the amount of stellar mass and thus also infer the amount of dark matter in these systems.

Studying the amount of dark matter contribution, one could in principle make a good estimation of the stellar mass-to-light ratio.

Mass-to-light ratios of early-type galaxies are of particular interest to understand the tilt of the fundamental plane. Virial relations imply that the effective surface brightness $I_{\text{eff}}$, the effective radius $r_{\text{eff}}$ and the central velocity dispersion $\sigma_{0}$ in hot stellar systems are not independent of each other. This is revealed by the fundamental plane of early type galaxies.

The Salpeter IMF implies more low-mass stars and a higher mass-to-light ratio. In the R-band the scaling between the two cases is $\Upsilon_{\text{Salp}}\approx 1.56\Upsilon_{\text{Krou}}$

For galaxies that are far away, it is impossible to make star counts, for this reason, the mass to light ratio of the stellar population provides a simple constraint on the IMF (Russell J. Smith and John R. Lucey) 

Strong gravitational lensing of background galaxies provides a useful method to determine masses in elliptical galaxies, since it is difficult to constraint the IMF via $M_{\star}/L$

massive galaxies - Salpeter is a good IMF

A Koupra IMF finds a value of $\Upsilon$ of around 4 for the mass to light ratio. (R. J. Smith 2014) 

DM fraction in comparison with the IMF 

Studying the matter distribution given by strong gravitational lensing can give us information about the iMF of the BCGs

percentage of dark matter will allow me to define the IMF more precisely. I want to see what fraction of the mass, what fraction of the surface density is stars. 

At very small radii stars dominate the lensing mass, so that lensing provides a direct probe of the stellar mass-to-light ratio, with only small corrections needed for dark matter (Russell Smith and John R. Lucey 2013)

Salpeter is heavier than Kroupa 

Salpeter mass function is $n(M)\propto M^{-2.3}$ 

For spiral galaxies, the most used IMFs are Chabrier or Kroupa, but for elliptical galaxies, constraining the IMF via $M_{\star}/L$ poses a greater challenge since masses are more difficult to establish for dynamically-hot systems. This is where gravitational lensing plays an important role.

As said before, bulges appear to have heavier IMFs than disks (Dutton et. al 2013)

"Large M/L ratios could arise either from an excess of faint dwarf stars in a "bottom heavy" IMF, or from an excess of dark remnants in a "top heavy" IMF" (Russell J. Smith and John R. Lucey 2013).

Modelling the lensing configuration provides the total projection mass within an aperture.

strong lensing at different radii is useful.

if I got to certain radius I will have more dark matter, because light drops quickly. 

basically find how much dark matter and hoy many stars are there in the profile

Two interesting questions about the BCGs:

1. Where are they located

2. What is their stellar populations


\newpage
