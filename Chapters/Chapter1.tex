\chapter{Introduction}

The history of formation of stars in galaxies is a key topic in the understanding of galaxies themselves since it determines most of the physical processes of formation and evolution of the building blocks of our Universe. Understanding the way stars form allows us to understand those physical properties thus providing a useful framework on which to build a more elaborate theory of their subsequent evolution. We might have good ideas and some general agreement in the basics of formation of stars in galaxies, but our observational limitations don't allow us to say much about distant objects which would allow us to make a more elaborate and complete theory, so in principle, we can't assume that the stars have the same formation history in every galaxy and for every epoch of the Universe. The gas that forms stars might or might not create the same mixture of stars in every stellar system so it is important to see under what conditions we could assume a general trend and what implications in our observations this may have. 

For galaxies that are far away, it is impossible to make star counts with our current technology, for this reason, their mass to light ratio (product of their stellar populations) provides a simple constraint on the initial mass function (IMF) (which is a very fundamental and important quantity in the study of stellar systems because it constraints the physics of star formation but also because it allows us to infer stellar masses through observed luminosities.) as  discussed by Russell J. Smith \& John R. Lucey, 2013. But this is not always the case, in more complex and dense systems like the brightest cluster galaxies in a galaxy clusters (for which we don't properly understand their formation) we may have different IMFs than the ones associated with galaxies that are being formed now. These objects have a very old origin because although their build up and morphological formation is recent, their stellar populations are old and they correspond to the highest density peak, so it is difficult to relate their stellar populations accurately.

Everything we know from galaxy evolution is implicitly assuming an explicit form of the IMF, with very little variations since it is the method we use to connect evolutionary sequences, this of course, given the fact that if every galaxy had its own IMF then it would be too difficult to study their evolution because of the lack of any knowledge about its history. In the case of spiral galaxies for example, the most commonly used IMFs are Chabrier or Kroupa which are decently constrained given the facilities of our observations in our own galaxy, but for elliptical galaxies, constraining the IMF via $\textrm{M}_{\star}/\textrm{L}$ poses a greater challenge since masses are more difficult to establish for dynamically-hot systems. We have some clues and hints, for example bulges appear to have heavier IMFs than disks as mentioned by (Dutton et. al 2013).

So we see that in the context of the evolution of galaxies, there are many things that come together at the very heart of cosmology but also in the context of the stellar astrophysics and they need to be consistent with each other. Addressing this problem is not simple whatsoever, because we must take into account the non-baryonic matter that is present in the objects of study which affects the mass to light ratio determination. We have to take dark matter into consideration since it accounts for most of the dynamical mass of galaxies and it's the dominant contribution in most of their spacial scales. The problem would be much easier to study if we only had the stellar mass because the light measurements would be enough to constrain the stellar population, their evolution and their mass contribution. Being able to calculate the percentage of dark matter allow us to define the IMF more precisely. We want to see what fraction of the surface density is stars.

So, given the difficulty of this problem, various techniques have been developed to try to disentangle the mechanisms that drive the formation and evolution of galaxies regarding their stellar populations. One of them is by using gravitational lensing (Treu et. al. 2010) of background galaxies. Modelling the lensing configuration on a BCG provides a useful method to determine stellar and dark matter mass contribution in elliptical galaxies, since it is difficult to constraint the IMF via $\textrm{M}_{\star}/\text{L}$. Finding strong lensing in these systems can also give us information about the location of the mass center of the cluster through the lensing they produce. We usually assume that the centre of galaxy clusters lies in the BCG (George et. al. 2012) but the real position of the centre in galaxy clusters is still an unsolved problem (Harvey et. al, 2017). 

In this project we work with galaxy clusters that might be in the right range to search for gravitational lensing, and we have deep data that allows us to try to understand the questions just mentioned. We focus on the brightest cluster galaxy since it is a very massive system that could lens background objects and because photometry measurements can be made very accurately on them in comparison with their neighbouring galaxies. Although removing the BCG with theoretical models is quite difficult, it is still a good target to look at. 
   
\newpage
