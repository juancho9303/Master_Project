\chapter{Introduction}

The formation of stars in galaxies is a broad topic of study in cosmology and strophyisics since it has to do with all the process of formation and evolution of the building blocks of our universe. Their formation allows us to understand the initial conditions of the galaxies at the time they formed thus providing a useful framework on which to bbuild a more elaborate theory of their subsequent evolution. We might have good clues and agreement in the formation of stars in galaxies, but our observational limitations dont alow us to say much about distant objects so in principle, we don't have precise data that allows us to conclue that their formation occurs in the same way everywhere. For galaxies that are far away, it is impossible to make star counts, for this reason, the mass to light ratio of the stellar population provides a simple constraint on the IMF (Russell J. Smith and John R. Lucey). The gas that forms stars might or might not create the same mixture of stars that we are able to observe in our neighbourhood. If that is not the case, we might argue that dense regions like BCGs in Galaxy Clusters have a very old origin. Although their build up and morphological formation is recent, their stellar populations are old and they correspond to the highest density peak (citation), so maybe they form from stars so maybe their IMF (a very fundamental and important quantity in the study of stellar systems because that constraints the physics of star formation but also because it allows us to infer stellar masses through observed luminosities.) is different from a galaxy that is formed now (as they involve different metallitity for example) this is actually a very important question because if we don't properly understand their formation then it's difficult to relate the stellar populations. 

Everything we know from galaxy evolution is implicitely assuming an IMF, with very subtle variations because it is the method we use to conect evolutionary sequences, this of course, given the fact that if every galaxy had its own IMF it would be pretty much impossible to study their evolution because you lack any knowledge about its history and you can't assume an IMF anymore. We assume for example a general behavior for spiral galaxies, for which the most used IMFs are Chabrier or Kroupa, but for elliptical galaxies, constraining the IMF via $\textrm{M}_{\star}/\textrm{L}$ poses a greater challenge since masses are more difficult to establish for dynamically-hot systems. We have some clues and hints, for example bulges appear to have heavier IMFs than disks (Dutton et. al 2013)

So we see that in the context of the evolution of galaxies, there are many things that come together at the very heart of cosmology but also in the context of the study of the initial mass function (what stellar populations do the galaxies start with) that they have in their formation process so it is also relevant to take into account.

Adressing this problem is not simple whatsoever, because we must take into account the non-baryonic matter that is present in the objects we observe since their spatial scales involve a very important fraction (if not dominant) of the stellar systems. The problem would be much easier to study if we only had the stellar mass because the light measurements would be enough to constrain the stellar population, their evolution and their mass contribution. Being able to calculate the percentage of dark matter allow us to define the IMF more precisely. We want to see what fraction of the surface density is stars.

So, given the difficulty of this problem, various techiniques have been developed to try to disentangle the mechanisms that drive the formation and evolution of galaxies regarding their stellar populations. 

One of them is using gravitational lensing of background galaxies because modelling the lensing configuration provides the total projection mass within an aperture. It gives very accurate information about the total mass enclosed by a lens galaxy like a BCG thus constraining even more the range of masses of the stars and dark matter. Strong gravitational lensing of background galaxies provides a useful method to determine masses in elliptical galaxies, since it is difficult to constraint the IMF via $textrm{M}_{\star}/\text{L}$. Finding strong lensing in these systems can also give us information about the location of the BCGs (citation) since theay are complex systems that often undergo strong interactions that we can observe through the lensing they produce. 

In this project we work with galaxy clusters that might be in the right range to search for gravitational lensing, and we have deep data that allows us to try to understand the questions just mentioned. We focus on the brightest cluster galaxy since it is a very massive system that could lens background objects and because photometry measurements can be made very accurately on them in comaprison with their neighbouring galaxies. Although removing the BCG with theoretical models is quite difficult, it is still a good target to look at. 

The plot of the enclosed mass shows what radial scale we need to prove. One way could be through dynamics, another could be through gravitational lensing (talk about lensing). If we take different IMFs, how sensitive is the matter content to the choice of IMF?. Smaller the dark matter contribution, the smaller the overall error you make since light is easier to constraint and thus the IMF. Strong lensing measures exactly the enclosed mass so we need to know how much of its contribution we need to substract, the less we have to substract, the better for the determination of the IMF. If the effect of the IMF is very subtle in the mass vs radius plot, then we would need to know the dark matter distribution very well, but if the effect of the IMF is not very subtle, the less you need to know about the dark matter distribution. A recent sudy of a BCG mentions the relevance of this spatial scale, at very small radii stars dominate the lensing mass, so that lensing provides a direct probe of the stellar mass-to-light ratio, with only small corrections needed for dark matter (Russell Smith and John R. Lucey 2013) 

That's why the IMF is relevant, because it allows to see how much it moves up and down. If you look at a galaxy, why is it not possible to get the mass to light ratio? This is because the dark matter is more diluted than stellar light so the mass follows light behavior is not valid and a well understood theory of the dark matter halos has to be taken into account, NFW (Navarro, Frenk \& White,1996) provided a very consistent model for dark matter halos using N-body simulations so we can relate the lensing of the halos given this NFW density profile and putting special atention in the spacial scales on which the dark matter is relevant and where it starts to be the dominant contribution of the system and thus the lensing. 
  
\newpage
