\chapter{Introduction}

Stellar mass to light ratio and Stellar populations in the BCGs (ac'a es donde se deben hacer las citaciones mas importantes)

-Galaxy Clusters-

IMF is a very fundamental and important quantity in the study of stellar systems because it constraints the physicis of star formation but also because it allows us to infer stellar masses through observed luminosities.

the correct use of an IMF in the context of gravitational lensing on masssive objects like early type galaxies in galaxy clusters can help us constraint the ammount of stellar mass and thus also infer the ammount of dark matter in these systems.

Studying the ammount of dark matter contribution, one could in principle make a good estimation of the stellar mass'to'light ratio.

For galaxies that are far away, it is impossible to make star counts, for this reason, the mass to light ratio of the stellar population provides a simple constraint on the IMF (Russell J. Smith and John R. Lucey) 

Strong gravitational lensing of background galaxies provides a useful method to determinemasses in elliptical galaxies, since it is difficult to constrint the IMF via M/L

massive galaxies - salpeter is a good IMF

A Koupra IMF finds a value of gamma of around 4 for the mass to light ratio. (R. J. Smith 2014) 

DM fraction in comparison with the IMF 

Studying the matter distribution given by strong gravitational lensing can give us informarion about the iMF of the BCGs

percentage of dark matter will allow me to define the IMF more precisely. I want to see what fraction of the mass, what fraction of the surface density is stars

strong lensing at different radii is usefull.

if I got to certeain radius I will have more dark matter, becaue light drops quickly. 

basically find how much dark matter and hoy many stars are there in the profile


\newpage
