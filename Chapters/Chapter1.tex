\chapter{Introduction}

Understanding the formation history of stars allows us to comprehend many physical properties of their host galaxies thus providing a useful framework on which to build a more elaborate theory of their subsequent evolution. We might have good ideas and some general agreement in the basics of formation of stars in galaxies, but our observational limitations don't allow us to say much about distant objects which we need to make a more elaborate and complete theory. In principle, we can't assume that all populations of stars have the same formation history in every galaxy and for every epoch of the Universe. The molecular clouds (dense concentrations of interstellar gas and dust) that collapse gravitationally to form stars might or might not create the same mixture of stars in every stellar system since this depends strongly on their composition and the environment in which they collapse, so it is important to see under what conditions we could assume a general trend and what implications in our observations this may have.

For galaxies that are far away, it is impossible to make individual star counts with our current technology, for this reason, their mass-to-light-ratio $\Upsilon$ (which depends on their stellar populations) provides a simple constraint on their number of stars per unit mass given by the initial mass function (IMF), which is a very fundamental and important quantity in the study of stellar systems because it constraints the physics of star formation but also because it allows us to infer stellar masses through observed luminosities as discussed by \textcolor{blue}{Smith \& Lucey} (\citeyear{Reference7}). Everything we know from galaxy evolution is implicitly assuming a certain form of the IMF with very little variations since it is the method we use to connect evolutionary sequences, this of course, given the fact that if every galaxy had its own IMF then it would be too difficult to study their evolution because of the lack of any knowledge about their history. 

For stars, measurements of the luminosity function can be used to derive the Initial Mass Function (IMF). For galaxies, this is more difficult because mass to light ratio (M/L) of the stellar population depends upon the star formation history of the galaxy. A satisfactory determination of the IMF in galaxies is a difficult task, since it depends on very reliable and well-understood data and calibrations and because is seems to be intimately dependent on
the galaxy’s formation history (\textcolor{blue}{Cappellari et al.} \citeyear{Reference19}). The determination is usually made by observing star counts, getting the present-day luminosity function (assuming proper mass-luminosity relation and theoretical models that take into account the metallicity), getting the present-day mass function (for some evolutionary tracks and metallicities) and assuming some star formation history to get the IMF. All these assumptions, and the fact that we have to extrapolate these results to systems in which we can't get star counts make the determination very complicated. Moreover, another difficulty of the determination of IMF is that the classical assumption of a single IMF covering the whole mass range is being questioned in favour of a multiple-component IMF to account for possible different formation modes.

Despite these difficulties, we have some observational information about IMFs in galaxies. In the case of spiral galaxies for example, the most commonly used IMFs are Chabrier (\textcolor{blue}{Chabrier} \citeyear{Reference31}) and Kroupa (\textcolor{blue}{Kroupa} \citeyear{Reference30}). With Kroupa favouring a higher number of stars in the low mass regime ($\textrm{M}<0.5$ $\textrm{M}_{\odot}$) as compared to Chabrier's. These IMFs are well constrained given the facilities of our observations in our own galaxy. Also, several studies such as the one made by \textcolor{blue}{Brewer et al.} (\citeyear{Reference16}) suggest that bulges have heavier IMFs than disks as in the case of the commonly used Salpeter IMF by \textcolor{blue}{Salpeter} (\citeyear{Reference33}). 

Although these primary assumptions given by our limited observational evidence might not be too far from reality, we must note that when we study more complex and dense systems like the brightest cluster galaxies (BCG) in galaxy clusters or giant elliptical galaxies in general, constraining the IMF via $\textrm{M}/\textrm{L}$ might be way more complex and poses a greater challenge since masses are more difficult to establish for dynamically-hot systems like them. Measuring $\Upsilon$ in these systems is not a truly accurate constraint on the IMF since we may have different stellar formation histories than the ones associated with galaxies that are being formed now. These objects have a very old origin (although their build up and morphological formation is recent) because their stellar populations are old and they correspond to the highest density peak, so it is difficult to relate their stellar populations accurately.

The $\text{M}/L$ depends on galaxy type, but due to the lack of multi-wavelength photometry, it is often assumed that all cluster galaxies are composed of the same stellar population. If one assumes an old stellar population (and therefore a high $\text{M}/L$), the mass of the late-type galaxies (and thus the cluster as a whole) is over-estimated (\textcolor{blue}{Van der Burg et al.} \citeyear{Reference2}).

Mass-to-light ratios of early-type galaxies are of particular interest to understand the tilt of the fundamental plane. Virial relations imply that the effective surface brightness $I_{\text{eff}}$, the effective radius $r_{\text{eff}}$ and the central velocity dispersion $\sigma_{0}$ in hot stellar systems are not independent of each other. This is revealed by the fundamental plane of early type galaxies.

This general view shows that when dealing with the evolution of galaxies, there are many things that come together in the context of cosmology but also in the context of stellar astrophysics and they need to be consistent with each other. Additional complications need to be considered when dealing with massive ellipticals and BCGs, one of them is that they have a strong dependency on their non-baryonic matter content which affects the mass-to-light-ratio determination. This dark matter contribution accounts for most of the dynamical mass of galaxies and it's the dominant contribution in most of their spacial scales, specially in the outer regions. The problem would be much easier to study if we only had the stellar mass because the light measurements would be enough to constrain the stellar populations, their evolution and their mass distribution. 

Being able to calculate the percentage of dark matter allows us to define the IMF more precisely. So we want to see what fraction of the surface density is given by stars and what are the spatial scales in which DM becomes the dominant contribution to the enclosed mass. DM halos seem to have a diluted profile in comparison to the stellar content of galaxies (\textcolor{blue}{Navarro Frenk \& White} \citeyear{Reference17}) so, in principle, there is a region near the center of these massive systems in which the stellar mass is the dominant contribution. This implies that accurate measurements of their luminosity could give precise determinations of their mass to light ratio thus giving us some knowledge of their IMFs.

Recent studies have investigated how the IMF varies with galaxy mass, specially in elliptical galaxies. One of the methods used for this study is a rather indirect method, where galaxy stellar masses are determined from stellar population synthesis models that actually do not resolve the IMF, the results suggest that lower mass early-type galaxies (with dispersions $\sigma \approx 200$ km/s) seem to be consistent with a Milky-Way type IMF (e.g. a Kroupa or Chabrier IMF). In high-dispersion elliptical galaxies, however, stellar mass-to-light ratios are about a factor of 2 times higher than expected from a Kroupa IMF. Some studies indicate that the IMF in massive galaxies seems to be more dwarf dominated than in the Milky-Way so that they can be described by a Salpeter IMF (\textcolor{blue}{Thomas} \citeyear{Reference28}). Also, bulges appear to have heavier IMFs than disks. . Figure [1.1] shows the dependence of the enclosed mass of an elliptical galaxy on different IMFs (for a constant mass-to-light ratio).

\begin{figure}[H]
\centering
\includegraphics[width=12cm]{images/Enclosed_Mass_IMFs.png}
\caption[Enclosed mass for different IMFs in a galaxy]{Enclosed mass for different IMFs in a galaxy of $\textrm{M}_{200}\approx 10^{12}$ $\textrm{M}_{\odot}$. The figure illustrates the difference of the enclosed mass profle for different assumptions of the IMF in the galaxy. A heavier IMF like Salpeter will yield a higher mass profile with a significant difference from the Chabrier and Kroupa IMFs which have a very subtle difference between them. This profiles are made by integrating the de Vaucouleurs' surface brightness profile and using the mass-to-light ratio for every IMF with the assumption that mass follows light.}
\end{figure}   

As seen in Figure [1.1], the initial mass distribution of galaxies (and consequently their future stellar populations) provides a very powerful constraint on the mass of galaxies so their proper determination is very important. Various techniques have been developed to try to understand the stellar populations (consequence of the IMF) that form these massive systems. One of them is by using gravitational lensing  of  background galaxies (\textcolor{blue}{Treu et al.} \citeyear{Reference1}). Modelling the lensing configuration on a massive galaxy like a BCG provides a useful method to determine stellar and dark matter mass contributions in elliptical galaxies, since it is difficult to constraint the IMF via $\textrm{M}/\text{L}$ as mentioned before. 

Strong lensing measures exactly the total enclosed mass so we need to know how much of the dark matter contribution we need to subtract and on what scales this substraction is important, the less we have to subtract, the better for the determination of the IMF. If the effect of the IMF is very subtle in the mass vs radius plot, then we would need to know the dark matter distribution very well, but if the effect of the IMF is not very subtle, the less we need to know about the dark matter distribution. 

An example of the relevance of the spatial scales is the study of a giant elliptical galaxy (ESO325-G004) by \textcolor{blue}{Smith \& Lucey} (\citeyear{Reference7}), which shows that at very small radii stars dominate the lensing mass, so that lensing provides a direct probe of the stellar mass-to-light ratio at these scales, with only small corrections needed for dark matter.  

Moreover, a more recent study by \textcolor{blue}{Smith et al.} (\citeyear{Reference34}) characterized the mass profile of the strong lensing BCG of a galaxy cluster (Abell 1201 located at $z=0.169$) and found that its proper mass modelling needs to include the component of an important central mass which could either be a supermassive black hole or a consequence of a non-uniform stellar mass to light ratio (from a bottom heavy IMF) affecting only the innermost part of the galaxy. His mass modellling is shown in Figure [1.2].

\begin{figure}[H]
\centering
\includegraphics[width=12cm]{images/smith.png}
\caption[Projected mass of cluster Abell 1201]{Cumulative projected mass profile for galaxy cluster Abell 1201 by \textcolor{blue}{Smith et al.} (\citeyear{Reference34}). The total (solid black) and component (solid colour) indicate the mass profiles including the center mass (NFW+Stars+M$_{\text{cen}}$). The dashed lines show the best fitting NFW+Stars model without the center mass.}
\end{figure}

The results of Smith's papers exemplify the importance of the determination of the scales on which strong lensing is useful to characterize the IMF and what scale differences we might encounter when dealing with massive early type galaxies or the characteristic BCGs of galaxy clusters (the latter being one of the main theoretical objectives of this project). 

Because lensing gives a very accurate estimate of the total enclosed mass, a lensed system in the core of the galaxy cluster would help us undestand the ammount of dark matter in that region and could also be interesting for a better determination of DM models.

Finding strong lensing in these systems can also give us information about the location of the mass center of the cluster through the lensing they produce. We usually assume that the centre of galaxy clusters lies in the BCGs (\textcolor{blue}{George et al.} \citeyear{Reference18}) but the real position of the centre in galaxy clusters is still an unsolved problem (\textcolor{blue}{Harvey et al.} \citeyear{Reference13}). 

The obervational analysis in this project is done for galaxy clusters that might be in the right range to search for gravitational lensing in the inner regions. We use deep data from CFHT in the $g,r,i,U$ bands that allows us to search for interesting targets (lensed systems) and probe the relevant spacial scales. We focus on the brightest cluster galaxy since it is a very massive system that could lens background objects and because photometry measurements can be made very accurately on them in comparison with their neighbouring galaxies. 

We subtract the light of the BCGs and search for galaxies that have been strongly lensed (multiple images and arcs) near the radial distance where we expect to find those systems. At the same time, we get several constraints on the ammount of objects that we expect to see and over which scales we expect to see them in our data by analyzing the predictions of the theoretical models and by conducting a statistical analysis of extensive data of galaxies at different redshifts. The observational procedures and the search for multiple images using photometric redshifts allow us to corroborate the expected number of lensing candidates.    

