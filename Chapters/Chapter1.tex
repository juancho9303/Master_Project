\chapter{Introduction}

Stellar mass to light ratio and Stellar populations in the BCGs

Galaxy Clusters

IMF is a very fundamental and important quantity in the study of stellar systems because it constraints the physics of star formation but also because it allows us to infer stellar masses through observed luminosities.

the correct use of an IMF in the context of gravitational lensing on massive objects like early type galaxies in galaxy clusters can help us constraint the amount of stellar mass and thus also infer the amount of dark matter in these systems.

Studying the amount of dark matter contribution, one could in principle make a good estimation of the stellar mass-to-light ratio.

Mass-to-light ratios of early-type galaxies are of particular interest to understand the tilt of the fundamental plane. Virial relations imply that the effective surface brightness $I_{\text{eff}}$, the effective radius $r_{\text{eff}}$ and the central velocity dispersion $\sigma_{0}$ in hot stellar systems are not independent of each other. This is revealed by the fundamental plane of early type galaxies.

The Salpeter IMF implies more low-mass stars and a higher mass-to-light ratio. In the R-band the scaling between the two cases is $\Upsilon_{\text{Salp}}\approx 1.56\Upsilon_{\text{Krou}}$

For galaxies that are far away, it is impossible to make star counts, for this reason, the mass to light ratio of the stellar population provides a simple constraint on the IMF (Russell J. Smith and John R. Lucey) 

Strong gravitational lensing of background galaxies provides a useful method to determine masses in elliptical galaxies, since it is difficult to constraint the IMF via $M_{\star}/L$

massive galaxies - Salpeter is a good IMF

A Koupra IMF finds a value of $\Upsilon$ of around 4 for the mass to light ratio. (R. J. Smith 2014) 

DM fraction in comparison with the IMF 

Studying the matter distribution given by strong gravitational lensing can give us information about the iMF of the BCGs

percentage of dark matter will allow me to define the IMF more precisely. I want to see what fraction of the mass, what fraction of the surface density is stars. 

At very small radii stars dominate the lensing mass, so that lensing provides a direct probe of the stellar mass-to-light ratio, with only small corrections needed for dark matter (Russell Smith and John R. Lucey 2013)

Salpeter is heavier than Kroupa 

Salpeter mass function is $n(M)\propto M^{-2.3}$ 

For spiral galaxies, the most used IMFs are Chabrier or Kroupa, but for elliptical galaxies, constraining the IMF via $M_{\star}/L$ poses a greater challenge since masses are more difficult to establish for dynamically-hot systems. This is where gravitational lensing plays an important role.

As said before, bulges appear to have heavier IMFs than disks (Dutton et. al 2013)

"Large M/L ratios could arise either from an excess of faint dwarf stars in a "bottom heavy" IMF, or from an excess of dark remnants in a "top heavy" IMF" (Russell J. Smith and John R. Lucey 2013).

Modelling the lensing configuration provides the total projection mass within an aperture.

strong lensing at different radii is useful.

The not so direct connection between the NFW profile of gravitational lensing, the constraints we put on the stelllar poplations if we want to measure background galaxies, 

if I got to certain radius I will have more dark matter, because light drops quickly. 

basically find how much dark matter and hoy many stars are there in the profile

Two interesting questions about the BCGs:

1. Where are they located

2. What is their stellar populations

Henk: First parragraph is very general, and it gets narrower until the end where you say "this is what I want to do". How do stars form? is it the same everywhere? If I start with a clump of gas do I get the same mix of stars? If it's not the case, I might expect that maybe earlier on (people suspect and it's not unreasonable to think) that the origin of the BCG is old, their build up is recent but their stellar populations are old and they correspond to the highest density peak so perhaps they form from stars so maybe their imf is different from let's say a galaxy that is formed now (different metallicity etc) so that's an important question becuase if we don't know this, then it's very difficult to relate stellar populations. Everything we now know about galaxy evolution is implicitely assuming this IMF, not varying it too much, that's how they can connect evolutionary sequences, of course, if every galaxy could do whatever it wants then it becomes difficult because now you see a galaxy and you have no knowledge anymore about its history. So talk about stellar populations, what do you see, you see the combination between the evolution of the galaxies, there are many things that come together but at the heart it's also the IMF (what population do I start with). This is a fundamental question. It adresses the point, it is quite difficult to do this because we have dark matter, it would be much easier if we only had the stellar mass because you would have only to measure the light and you would be done. Then you can put in more specific things, like how would you try to disentangle this question, you can use lensing for example. (If you find strong lensing you could also adress the question of where the BCGs are located) so it is useful to mention that this might be a good question that our project could address. So mention the main motivation but also mention the posibilities that it brings, it is always better to solve two problems instead of one. It's always nice to look for things in your data that were not part of the main motivation of the data. This data for example was usefull for measuring number of stars in the sky since it has random locations so it provides a good sample to look for data like that.
We are looking at galaxy clusters that might be in the right range to serach for gravitational lensing. even though it is hard to get rid of the BCG, it is an interesting target to look at. The plot of the enclosed mass shows what radial scale we need to prove, so one way would be through dynamics and the other one would be throguh gravitational lensing (that's how you introduce the lensing). If I take different IMFs, how much does the mass vs radius plot move up or down. Smaller the dark matter contribution, the smaller the overall error you make since light is easier to constraint and also the IMF. Strong lensing measures exactly the mass so the question is how much you have to substract, so the less you have to substract, the better. If the effect of the IMF is very subtle, then you need to know the dark matter contribution really really well, if its not so subtle you dont need to knnow the dark matter distribution very well. that's when the IMF comes in, you want to know, given some choice for stellar mass, we want to see how much it moves up or down to see what results you get. If I look at a galaxy, why can I cont get the mass to light ratio?, there you introduce the dark matter haloes, that's where all comes together and there you say there must be a scale where the stars dominate, that's the scale I want to measure, but how do I measure that scale? there you have a very nice story, you conected the problem and the approach, then your third chapter is how do you do this in practice. 


\newpage
